\documentclass[12pt]{article}
\usepackage[warn]{mathtext}
\usepackage[T2A]{fontenc}
\usepackage[utf8]{inputenc}
\usepackage[russian]{babel}
%\usepackage{cite}
%\usepackage{amsfonts}
%\usepackage{lineno}
%\usepackage{hyperref}
%\usepackage{subfig}
%\usepackage{graphicx}
%\usepackage{xcolor}
%\usepackage{graphicx}
%\usepackage{bm}
%\DeclareGraphicsExtensions{.pdf,.png,.jpg}

%\usepackage{color}

\author{Карцев Вадим, Б01-904}            %Edit author here to change Article author on title page
\title{Обработка результатов измерений}
%Date of article is compilation date
%If you want to set another date, remove % in line below
%\date{1 сентября 2017 г.}

\begin{document}

  \maketitle

  \newpage

  \section{Измерения и погрешности}

    \subsection{Результат измерения}

      Очевидно, что когда мы измеряем некоторую величину, имеет место быть некоторая
      неточность измерений. Например, измеряя длину тела линейкой, мы можем столкнуться
      с тем, что линейка может быть неточно положена, иметь неточные деления.
      Даже если добиться точности расположения линейки, все равно имеет место быть округление,
      так как деления линейки имеют некоторую цену. У устройств без шкалы на дисплее
      все равно может быть отображено только конечное число цифр после запятой.
      Таким образом, то, что мы называем измерением - это некоторое \textit{идеализированное значение},
      только приближенное к реальному.

      Назовем погрешностью измерения разницу между измеренным и «истинным» значениями

      $$
        \delta x = x_{изм} - x_{ист}
      $$

      Однако величину $\delta x$ невозможно точно определить ввиду невозможности
      узнать истинное значения некоторой величины.

      О каких-либо величинах принято говорить не как о точных значениях, а скорее
      как о некотором промежутке

      $$
        x = x_{изм} \pm \delta x
      $$

      Кроме этого часто для наглядности используют относительную погрешность

      $$
        \varepsilon_x = \frac{\delta x}{x_{изм}}
      $$

    \subsection{Многократные измерения}

      Если мы несколько раз измерим одну и ту же величину, вероятно мы получим
      расходящиеся по значению результаты.

      $$
        \textbf{X} = \{x_1, x_2, ..., x_n\}
      $$

      В таком случае результат измерений является случайной величиной, которую
      можно будет описать некоторым \textit{веротностным законом} - \textit{распределением}.
      Вычислим среднее значение величины по набору \textbf{X}

      \begin{equation}
        \langle x \rangle = \frac{x_1 + x_2 + ... + x_n}{n} \equiv
        \frac{1}{n} \sum_{i=1}^n x_i
      \end{equation}

      Так же мы будем орудовать понятием отклонения. Так, отклонение каждого значения
      от среднего это

      $$
        \Delta x_i = x_i - \langle x \rangle, \hspace{1cm} i = 1...n
      $$

      Разброс совокупности данных ${x_i}$ относительно среднего принято характеризовать
      \textit{среднеквадратичным отклонением}

      \begin{equation}
        s = \sqrt{\frac{\Delta x_1^2 + \Delta x_2^2 + ... + \Delta x_n^2}{n}} \equiv
        \sqrt{\frac{1}{n} \sum_{i=1}^n \Delta x_i^2}
      \end{equation}

      или кратко

      \begin{equation}
        s = \sqrt{\langle \Delta x^2 \rangle} \equiv \sqrt{\langle (x - \langle x \rangle)^2 \rangle}
      \end{equation}

      При устремлении $n$ к бесконечности и достаточном качестве метода измерений почти
      все отлонения $\delta x_i$ скомпенсируются и можно ожидать что среднее значение устремится
      к некоторому пределу

      $$
        \overline{x} = \lim_{n \rightarrow \infty} \frac{1}{n} \sum_{i=1}^n x_i
      $$

      Тогда полученное значение $\overline{x}$ можно считать «истиным» средним для исследуемой величины
      Предельную величину среднеквадратичного отклонения обозначим как

      $$
        \sigma = \lim_{n \rightarrow \infty} \sqrt{\frac{1}{n} \sum_{i=1}^n \Delta x_i^2}
      $$

      Итак, если набор значений имеет не слишком большой разброс, то можно с некоторой
      натяжкой считать, что $\langle x \rangle \approx \overline{x}$

    \subsection{Классификация погрешностей}

      Всегда нужно проводить несколько замеров величины в одинаковых условиях, чтобы
      убедиться в стабильности величины и правильности выбранного метода измерений.
      Иногда во время измерений возникают грубые ошибки - «промахи». Естественно,
      промахи не нужно учитывать при обработке данных. Однако, это может привести
      к потере данных или помешать открытию некоторого нового явления. Поэтому необходимо
      тщательно анализировать причины появления аномалий в данных.

      Погрешности можно разделить на \textit{систематические}, которые одинаково
      проявляются при множественных проведениях опыта и на \textit{случайные}, которые
      хаотичны как по величине так и по знаку.

      Так же можно разделить погрешности на

      \begin{itemize}
        \item \textit{инструментальные погрешности}, связанные с насовершенством конструкции
        или ошибками калибровки измерительных приборов;

        \item \textit{методические погрешности}, связанные с несовершенством теоретической
        модели явления или неточностью метода измерения;

        \item \textit{естественные погрешности}, которые связаны со случайным характером
        изменения физической величины. Зачастую они показывают природу некоторого
        явления, поэтому ими нельзя пренебрегать.
      \end{itemize}

      \subsubsection{Случайные погрешности}

        Большинству физических явлений присущ случайный характер. Случайную погрешность
        можно обнаружить при многократном повторении некоторого опыта. Если случайные
        отклонения с разными знаками прибилизительно равновероятны, то можно считать, что
        погрешность среднего значение $\langle x \rangle$ будет меньше, чем погрешность
        одного измерения.

        Случайные погрешности могут быть связаны с \textit{особенностями приборов},
        \textit{особенностями или несовершенством методики измерения},
        \textit{несовершенством объекта измерений} или \textit{случайным характером явления}.

        В последних двух случаях мы сами заменяем отдельные измерения средним значением.
        Таким образом мы можем потерять много иформации о объекте исследования и
        прежде чем отбрасывать случайную погрешность, необходимо убедиться, что
        погрешность вызвана приборами, а не характером объекта.

      \subsubsection{Систематические погрешности}

  \section{Элементы теории ошибок}

    \subsection{Случаная величина}

    \subsection{Нормальное рапределение}

    \subsection{Независимые величины}

    \subsection{Погрешность среднего}

    \subsection{Результирующая погрешность опыта}

    \subsection{Обработка косвенных измерений}

      \subsubsection{Случай одной переменной}

      \subsubsection{Случай многих переменных}

  \section{Рекомендации по выполнению и представлению результатов работы}

    \subsection{Проведение измерений}

      \subsubsection{Правила ведения лабораторного журнала}

      \subsubsection{Подготовка к работе}

      \subsubsection{Начало работы}

      \subsubsection{Выбор количества измерений}

      \subsubsection{Измерения}

      \subsubsection{Рассчёты, анализ и представление результатов}

    \subsection{Анализ инструментальных погрешностей}

    \subsection{Отчёт о работе}

      \subsubsection{Требования к содержанию разделов}

      \subsubsection{Правила округления}

    \subsection{Построение графиков}

      \subsubsection{Рекомендации по оформлению графиков}

    \subsection{Некоторые типичные ошибки обработки данных}

  \section{Оценка параметров}

    \subsection{Метод минимума хи-квадрат}

    \subsection{Метод максимального правдоподобия}

    \subsection{Метод наименьших квадратов}

    \subsection{Проверка качества аппроксимации}

    \subsection{Оценка погрешности параметров}

    \subsection{Метод построения наилучшей прямой}

      \subsubsection{Метод наименьших квадратов}

      \subsubsection{Погрешность МНК в линейной модели}

      \subsubsection{Метод хи-квадрат построения прямой}

      \subsubsection{Недостатки и условияя применимости методов}

\end{document}
